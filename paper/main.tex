\documentclass[
  aps,
  prapplied,
  twocolumn,
  superscriptaddress,
  nofootinbib
]{revtex4-2}

\usepackage{graphicx}
\usepackage{amsmath}
\usepackage{amssymb}
\usepackage{bm}
\usepackage{physics}
\usepackage{siunitx}
\usepackage{placeins}
\usepackage{dcolumn}
\usepackage{hyperref}

\begin{document}

\title{Zero-Shot Detection of Elastic Transient Morphology Across Physical Systems}

\author{Jose Sánchez Andreu}
\affiliation{Independent Researcher, Murcia, Spain}

\date{\today}
\begin{abstract}
We test whether a representation learned from interferometric strain transients in
gravitational-wave observatories can act as a frozen morphology-sensitive operator for
unseen sensors, provided the target signals preserve coherent elastic transient structure.
Using a neural encoder trained exclusively on non-Gaussian instrumental glitches, we perform
strict zero-shot anomaly analysis on rolling-element bearings without retraining, fine-tuning,
or target-domain labels.

On the IMS--NASA run-to-failure dataset, the operator yields a monotonic health index
$HI(t)=s_{0.99}(t)/\tau$ normalized to an early-life reference distribution, enabling fixed
false-alarm monitoring at $1-q=10^{-3}$ with $\tau=Q_{0.999}(\mathcal{P}_0)$.
In discrete fault regimes (CWRU), it achieves strong window-level discrimination
($\mathrm{AUC}_{\mathrm{win}}\approx 0.90$) and file-level separability approaching unity
($\mathrm{AUC}_{\mathrm{file}}\approx 0.99$). Electrically dominated vibration signals (VSB) show
weak, non-selective behavior, delineating a physical boundary for transfer.

Under a matched IMS controlled-split protocol, a generic EfficientNet-B0 encoder pretrained on
ImageNet collapses in the intermittent regime ($\Lambda_{\mathrm{tail}}\approx 2$), while the
interferometric operator retains strong extreme-event selectivity
($\Lambda_{\mathrm{tail}}\approx 860$), indicating that the effect is not a generic property of CNN features.
Controlled morphology-destruction transformations selectively degrade performance despite per-window normalization,
consistent with sensitivity to coherent time--frequency organization rather than marginal amplitude statistics.
\end{abstract}

\maketitle
\section{Introduction}
\label{sec:intro}

The detection and characterization of transient phenomena is central across
experimental physics. Short-duration deviations from nominal behavior often
signal nonlinear dynamics, changes in boundary conditions, or localized
dissipation not captured by stationary or time-averaged descriptions. In
precision interferometry, weak non-Gaussian strain transients must be
identified and mitigated to ensure reliable scientific inference; in rotating
mechanical systems, transient elastic responses driven by impacts, frictional
contacts, and incipient defects are primary indicators of structural
degradation and instability.

Despite vast differences in physical scale and sensing modality, both settings
share a common mechanism: elastic wave propagation in constrained media.
Material properties, geometric confinement, and boundary conditions jointly
govern dispersion, reflection, and the temporal organization of propagating
wave packets, so elastic transients are structured objects whose time--frequency
organization reflects the physics of the medium. In interferometric detectors,
such disturbances arise from suspension friction, scattered light, seismic
upconversion, and opto-mechanical instabilities; in rolling-element bearings,
analogous responses emerge from localized contact mechanics, surface defects,
and impulsive load transfer across raceways. While characteristic frequencies,
amplitudes, and units differ by orders of magnitude, both classes of systems
impose strong physical constraints that can organize transient energy coherently
across time and frequency. This motivates the hypothesis that certain
\emph{coherently organized time--frequency morphologies} of non-Gaussian elastic
transients may be preserved across physically distinct systems when propagation
remains elastic and constrained.

In gravitational-wave interferometers, non-Gaussian strain transients
(``glitches'') are a major detector-characterization challenge
\cite{powell2015classification,zevin2017gravityspy,cabero2019blip}. Prior work
has shown that glitch populations exhibit rich, repeatable time--frequency
morphologies that can be learned from data and exploited for monitoring and
data-quality assessment. Yet most transient-detection approaches remain strongly
domain-specific: gravitational-wave pipelines rely on detailed instrument
models, transfer functions, and heuristics, while mechanical condition
monitoring typically trains detectors on the target machinery, requiring labels,
retraining, or domain adaptation as operating conditions change. Both paradigms
tend to couple detection to absolute frequency content, signal energy, or
stationary descriptors, and can therefore miss nonstationary transient structure
governed by elastic propagation rather than static spectral content.

Here we test a different hypothesis: that a representation trained to encode
the \emph{morphological structure} of elastic transients can function as a fixed
physical operator across domains, independent of absolute scale and sensor
modality. We study a latent operator $\mathcal{F}$ trained exclusively on
interferometric strain transients and evaluate whether it exhibits selective
anomaly sensitivity when applied \emph{zero-shot} to rotating mechanical systems.
Rather than acting as a domain-specific classifier, $\mathcal{F}$ operates as a
morphology-sensitive projection from normalized time--frequency representations
into a latent space that encodes structural regularities associated with elastic
wave propagation.

This view is summarized schematically in Fig.~\ref{fig:concept} by treating the
trained representation as a fixed measurement operator acting on normalized
time--frequency structure.
\begin{figure}[t]
  \centering
  \includegraphics[width=\columnwidth]{figures/Fig1.pdf}
 \caption{
\textbf{Conceptual framework for cross-domain morphological analysis.}
A fixed latent operator $\mathcal{F}$, trained exclusively on interferometric
strain transients (source domain), maps normalized time--frequency
representations from multiple physical systems into a shared latent space
$\mathcal{Z}$. The operator acts as a morphology-sensitive projection that
suppresses dependence on absolute scale and sensor modality, while preserving
structural organization associated with elastic wave propagation in constrained
media. Zero-shot anomaly sensitivity emerges only when target-domain signals
preserve compatible elastic transient structure; systems lacking mechanically
mediated propagation do not exhibit separation.
}
  \label{fig:concept}
\end{figure}

We evaluate this hypothesis in both discrete fault regimes and progressive
degradation. Using the IMS--NASA run-to-failure bearing dataset, we test whether
the fixed operator $\mathcal{F}$ captures monotonic drift, yields an interpretable
health indicator, and enables early deviation detection under a controlled
false-alarm rate, without retraining or access to fault-domain labels. We also
evaluate separability on the CWRU bearing dataset under controlled fault
conditions, and use electrically dominated vibration signals as a negative
control to delineate the physical boundary of transfer. Classical treatments of
elastic and acoustic wave propagation in solids emphasize that transient
structure is governed primarily by material properties, geometry, and boundary
conditions rather than absolute scale \cite{graff1975wave,auld1973acoustic}.

To ensure falsifiability, we treat $\mathcal{F}$ as a fixed measurement device
and probe its response under controlled morphology-destruction transformations
(e.g., spectral filtering, temporal smearing, and reverberation-induced
self-interference). These interventions preserve marginal amplitude statistics
through per-window normalization while selectively disrupting time--frequency
organization. If sensitivity were driven by generic distributional shifts or
energy cues, performance would be invariant; transformation-dependent
degradation instead implicates preserved time--frequency structure rather than
amplitude-based or stationary statistical effects.

We emphasize that the reported invariance is empirical and
representation-induced, not a fundamental symmetry of the governing equations:
$\mathcal{F}$ does not encode a conserved quantity, but acts as a fixed
instrument whose response reveals shared structural regularities under the stated
protocol. Likewise, ``zero-shot'' refers here to an \emph{experimental deployment
protocol}: $\mathcal{F}$ encodes a prior learned solely from \emph{source-domain}
interferometric transient physics and is then held frozen; no target-domain
samples, labels, hyperparameter tuning, calibration, or adaptation are used to
shape the representation or the decision rule.

The remainder of this paper is organized as follows. Section~\ref{sec:methods}
describes the physical systems, datasets, and experimental protocol.
Section~\ref{sec:results} presents the run-to-failure analysis, zero-shot
transfer, and controlled degradation experiments. Section~\ref{sec:discussion}
discusses the physical interpretation and limitations, followed by concluding
remarks in Sec.~\ref{sec:conclusion}.
\section{Experimental setup and methods}
\label{sec:methods}
\subsection{Physical Systems and Datasets}
\label{sec:datasets}

The experiments reported in this work are designed to probe whether a fixed
latent representation, trained on interferometric strain transients, exhibits
selective sensitivity to elastic transient morphology across physically
unrelated systems. We therefore consider signals arising in constrained solid
media that span disparate spatial scales, sensing modalities, and operating
regimes, but share a common physical mechanism: elastic wave generation and
propagation governed by material properties, geometric confinement, and
boundary conditions.

By construction, all datasets are treated as passive measurement records.
No system-specific modeling, retraining, or domain adaptation is performed in
the target domains. The latent operator $\mathcal{F}$ is trained once on a
single source domain and subsequently applied as a fixed measurement device
under a strict zero-shot protocol. Details on source-domain training, preprocessing, and statistical scoring are provided in the Supplemental Material.

\paragraph{Source domain: Interferometric strain transients.}

The source domain consists of interferometric strain time series
$h(t)=\Delta L/L$ acquired from gravitational-wave detector instrumentation.
These data contain short-duration, non-Gaussian transients (instrumental
glitches) arising from suspension friction, seismic coupling, scattered light,
and opto-mechanical instabilities within complex optical assemblies. The
measured signals are broadband, nonstationary, and shaped by elastic wave
propagation through suspended and mechanically coupled components.

A dataset comprising $\mathcal{O}(10^{5})$ labeled transient events recorded
during nominal detector operation is used exclusively to train the latent
operator $\mathcal{F}$. No information from any mechanical target domain is used
during training. After training, $\mathcal{F}$ is frozen and applied without
modification in all subsequent experiments.

\paragraph{Target domain A: Run-to-failure degradation (IMS--NASA).}

The primary target domain is the IMS--NASA run-to-failure bearing dataset, which
records the full temporal evolution of rolling-element bearings from nominal
operation to catastrophic failure. Vibration signals are acquired via
accelerometers mounted on a rotating mechanical test rig, providing a
physically grounded record of progressive mechanical degradation driven by
localized contact mechanics and elastic wave propagation.
Figure~\ref{fig:ims_hi} summarizes the resulting Health Index trajectories for the
three IMS run-to-failure experiments under the fixed false-alarm protocol.
\begin{figure}[t]
  \centering
  \includegraphics[width=\columnwidth]{figures/Fig2.pdf}
  \caption{
  \textbf{Zero-shot run-to-failure monitoring on IMS--NASA.}
  Health Index trajectories for the three IMS bearing run-to-failure experiments,
  computed under a strict zero-shot protocol using a frozen operator trained on
  interferometric strain transients.
  The horizontal axis denotes record index (chronological order).
  For each record, window-level scores are aggregated via the $q=0.99$ quantile
  and normalized by a fixed nominal threshold $\tau$, defined as the
  $q=0.999$ quantile of the early-life reference distribution $\mathcal{P}_0$,
  corresponding to a nominal false-alarm probability of $10^{-3}$.
  The red dashed line marks $HI=1$ (nominal tail threshold) and the red marker
  indicates the first threshold crossing.
  }
  \label{fig:ims_hi}
\end{figure}
Unlike static fault benchmarks, this dataset enables the analysis of temporal
phenomena, including monotonic drift, early deviation from nominal behavior,
and the evolution of a continuous Health Index. We analyze three independent
run-to-failure trajectories (1st, 2nd, and 3rd tests), each exhibiting distinct
degradation dynamics. This dataset constitutes the central experimental
evidence for evaluating whether a fixed morphological operator can track
physical aging and incipient failure without retraining or access to fault
labels.

\paragraph{Target domain B: Discrete fault regimes (CWRU).}

As a complementary regime, we consider the Case Western Reserve University
(CWRU) bearing dataset, which provides vibration recordings under controlled
fault and no-fault conditions at steady operating points. 
The dataset is provided by the CWRU Bearing Data Center \cite{Smith2015CWRU}.
Unlike IMS, CWRU does not capture temporal degradation, but instead offers clean separation between nominal and faulty mechanical states.

Rolling-element bearing diagnostics and condition monitoring have a long
history in vibration analysis, where fault-induced elastic impacts excite
guided waves in raceways, rolling elements, and housings.
Early work established the importance of high-frequency resonance and impulsive
response for bearing fault detection \cite{mcfadden1984bearing}.
More recent benchmarks have formalized standard datasets and evaluation
protocols for vibration-based diagnostics
\cite{smith2015rolling,Randall2011}.

Performance on CWRU is reported at two resolutions: window-level
($\mathrm{AUC}_{\mathrm{win}}$) and file-level aggregation
($\mathrm{AUC}_{\mathrm{file}}$).

\paragraph{Negative control: Electrically dominated vibration (VSB).}

To delineate the physical domain of validity, we evaluate the same frozen operator on the
VSB power-line fault dataset, whose signals are dominated by narrowband electromagnetic
interference and incoherent sensor noise rather than mechanically mediated elastic-wave
propagation \cite{VSB2019}. If the reported transfer were driven by generic non-Gaussianity
or distributional shift, comparable separation would be expected in this regime; instead,
VSB serves as a negative control for physically specific elastic-transient morphology.

All datasets are processed using the identical time--frequency pipeline described in
Sec.~\ref{sec:tf_preprocessing}, enabling a strict zero-shot evaluation.
\begin{table}[b]
\caption{\label{tab:stft_params}
Signal processing parameters defining the time--frequency resolution.
The target sampling rate is chosen to align the elastic transient scale
with the receptive field of the interferometric operator.
}
\begin{ruledtabular}
\begin{tabular}{llll}
Parameter & Symbol & Value & Unit \\
\hline
Target sampling rate & $f_s$ & 4096 & Hz \\
Window size & $N_{\mathrm{win}}$ & 256 & samples \\
Hop length & $N_{\mathrm{hop}}$ & 128 & samples \\
FFT points & $N_{\mathrm{fft}}$ & 256 & bins \\
\hline
Temporal resolution & $\Delta t$ & 31.25 & ms \\
Spectral resolution & $\Delta f$ & 16.0 & Hz \\
Receptive field & $T_{\mathrm{RF}}$ & $\approx 1.5$ & s \\
\end{tabular}
\end{ruledtabular}
\end{table}

In the IMS run-to-failure dataset, each measurement file acquired at a given
time is treated as a \emph{record}. Each record is segmented into overlapping
windows, scored at the window level, and aggregated into record-level
statistics (e.g., upper quantiles) to produce a time-resolved Health Index
$HI(t)$.

All datasets use the fixed preprocessing pipeline in Sec.~\ref{sec:tf_preprocessing}.

\subsection{Time--Frequency Representation and Preprocessing}
\label{sec:tf_preprocessing}

All signals are projected onto a fixed time--frequency representation prior to
embedding. The latent operator $\mathcal{F}$ operates exclusively on
pre-generated spectrograms and does not access time-domain waveforms directly;
consequently, all time--frequency parameters are defined as part of the
preprocessing protocol and are held fixed across domains.
Signals are resampled to a common $f_s=4096\,\mathrm{Hz}$ to fix a shared time--frequency grid
and to match the transient time scale to the encoder’s effective receptive field.
The value is a convenient power-of-two choice for consistent STFT windowing; stronger
downsampling erases short elastic transients.
We compute Hann STFT spectrograms using $N_{\mathrm{win}}=256$, $N_{\mathrm{hop}}=128$,
and $N_{\mathrm{fft}}=256$, yielding $\Delta f = 16\,\mathrm{Hz}$ and $\Delta t = 31.25\,\mathrm{ms}$,
sufficient to resolve short elastic transients at the chosen grid.


All spectrograms are log-amplitude scaled and standardized independently on a
per-window basis. This normalization removes explicit dependence on absolute
signal amplitude, physical units, and sensor modality while preserving the
relative organization of energy across time and frequency. As a result, the
operator responds to dimensionless morphological structure rather than to
scale-dependent or stationary statistical features.
\subsection{Latent Morphological Operator and Anomaly Scoring}
\label{sec:operator_scoring}
Inputs to the frozen encoder are standardized log-amplitude spectrogram windows
constructed with a fixed preprocessing pipeline (Sec.~\ref{sec:tf_preprocessing}).
Per-window standardization suppresses absolute scale while preserving relative
time--frequency organization.
We emphasize that although the operator $\mathcal{F}$ is trained on a source domain,
it is not adapted, re-optimized, or re-calibrated in any way for the target domains.
All target-domain results therefore arise from the fixed geometry of the latent space
induced during source-domain training.
Throughout, $\mathcal{F}$ is treated as a fixed measurement instrument: after source-domain training, the encoder is frozen and used only to induce a geometry on normalized time--frequency inputs. The downstream anomaly score and threshold are defined independently from this geometry using nominal target-domain reference data, without any optimization against abnormal conditions.

\paragraph{Latent morphological operator.}

The latent operator $\mathcal{F}$ acts as a fixed measurement map from normalized
time--frequency structure to a latent representation,
\begin{equation}
\mathcal{F}: X \in \mathbb{R}^{M \times N} \;\mapsto\; z \in \mathcal{Z} \subset
\mathbb{R}^{d},
\end{equation}
$z$ empirically reflects structural regularities associated with elastic transient organization. 
In this work, $\mathcal{F}$ is a convolutional encoder that produces a pooled
$d=1280$ embedding. The classifier head used during source-domain training is
discarded, and the encoder is frozen for all target-domain experiments.

\paragraph{Score convention.}
Throughout this work we report the \emph{distance-form} Mahalanobis score
\begin{equation}
\mathcal{S}(z)=\sqrt{(z-\mu_0)^{\top}\Sigma_{\mathrm{LW}}^{-1}(z-\mu_0)}.
\label{eq:maha}
\end{equation}
All quantiles, detection thresholds, tail statistics, and Health Index definitions
are computed consistently on $\mathcal{S}$ in Eq.~(\ref{eq:maha}).
The corresponding quadratic form $\mathcal{S}^2(z)$ is related by
$\mathcal{S}(z)=\sqrt{\mathcal{S}^2(z)}$ and is not used elsewhere for thresholding,
aggregation, or reporting.
\paragraph{Aggregation levels and AUC metrics.}
We report performance at three complementary aggregation levels.
At the window level, $\mathrm{AUC}_{\mathrm{win}}$ evaluates instantaneous discrimination
between nominal and faulty windows.
At the file level, $\mathrm{AUC}_{\mathrm{file}}$ is computed by aggregating window scores
within each signal file using the arithmetic mean, reflecting diagnostic separability
when labels are associated with entire recordings.
For run-to-failure experiments, we report a record-level metric $\mathrm{AUC}_{\mathrm{rec}}$,
where each temporal record is summarized by a high quantile ($q=0.95$ or $0.99$) of its
window scores, enabling early detection under a fixed false-alarm threshold.

\paragraph{Anomaly scoring.}
Deviations from nominal behavior are quantified relative to a reference
distribution $\mathcal{P}_0$ constructed from embeddings associated with an
operational early-life period in the target domain. Given the high dimensionality
of the embedding space ($d=1280$) and the finite sample size of the reference set,
standard covariance estimation is ill-conditioned. 
More broadly, anomaly detection aims to identify deviations from nominal
behavior without explicit supervision on abnormal classes.
Comprehensive surveys have highlighted the diversity of statistical,
distance-based, and learning-based approaches to this problem
\cite{chandola2009anomaly}.
In high-dimensional settings, distance-based detectors in learned embedding
spaces have emerged as a practical compromise between interpretability and
expressivity \cite{ruff2018deeponeclass}.
We therefore employ the Ledoit--Wolf shrinkage estimator~\cite{Ledoit2004} to obtain a well-conditioned
covariance matrix $\Sigma_{\mathrm{LW}}$ and compute anomaly scores using Eq.~(\ref{eq:maha}).
We use the distance-form Mahalanobis score so that all thresholds and tail statistics are defined on a consistent scale.

\subsection{Run-to-Failure Analysis: Drift, Time-to-Detection, and Health Index}
\label{sec:ims_protocol}

To assess whether the latent operator $\mathcal{F}$ captures physically
meaningful degradation dynamics rather than static state separation, we apply
it to the IMS--NASA run-to-failure bearing dataset. Unlike discrete fault
benchmarks, this dataset provides a continuous temporal record from nominal
operation to catastrophic failure, enabling direct analysis of progressive
damage accumulation, early deviation, and long-term drift under realistic
operating conditions.
The IMS--NASA run-to-failure dataset provides full-life vibration recordings of
rolling-element bearings from nominal operation to catastrophic failure and is
widely used for studying degradation dynamics and prognostics.
The dataset was originally released through the NASA Ames Prognostics Data
Repository \cite{Qiu2006IMS}.

\paragraph{Baseline definition.}

A nominal reference distribution $\mathcal{P}_0$ is constructed from the first
$N$ records of each run, corresponding to early-life operation. Unless stated
otherwise, we use $N=100$; for IMS 2nd\_test, where an extended steady-state
segment is present, we use $N=200$. No data from later stages of the run are
used to define the baseline or detection thresholds.

\paragraph{Health Index.}

To obtain a time-resolved measure of system condition, window-level anomaly
scores are aggregated at the record level using upper quantiles of their
empirical distribution. For a record acquired at time $t$, we define the
dimensionless Health Index
\begin{equation}
HI(t)=\frac{s_q(t)}{\tau},
\end{equation}
where $s_q(t)$ is the $q$-th quantile of the window-level anomaly scores within
the record ($q=0.99$ unless otherwise specified), and $\tau$ is the fixed
detection threshold derived from the nominal reference distribution
$\mathcal{P}_0$.

By construction, $HI(t)=1$ corresponds to the extreme tail of nominal operation
under the chosen false-alarm convention, while $HI(t)>1$ indicates excess
occupancy of structurally anomalous transient events. This definition suppresses
sensitivity to shifts in central tendency and instead emphasizes the emergence
and persistence of rare, high-severity deviations in latent space, consistent
with extreme-event detection paradigms in experimental physics.

Normalization by $\tau$ renders the Health Index dimensionless and directly
comparable across runs, independent of absolute score scale or operating
conditions.

\paragraph{Time-to-Detection (TTD).}

We define the Time-to-Detection (TTD) as the earliest record at which the Health
Index persistently exceeds unity,
\begin{equation}
TTD = \min\left\{ t \;:\; HI(t) > 1 \;\; \text{for at least } K \text{ consecutive records} \right\},
\end{equation}
with $K=3$ unless otherwise stated. This persistence criterion suppresses
spurious threshold crossings due to isolated fluctuations and identifies the
earliest statistically significant deviation from the nominal elastic
transient morphology under a fixed false-alarm rate.

TTD is computed without access to failure labels, trend fitting, or
domain-specific thresholds, and therefore constitutes an unsupervised early
warning indicator derived entirely from the fixed operator response.

\paragraph{Drift analysis.}

To determine whether the temporal evolution of the Health Index reflects
systematic degradation rather than stochastic variability, we quantify
monotonic drift using Spearman’s rank correlation coefficient $\rho$ between
$HI(t)$ and record index. Spearman correlation is chosen to capture monotonic
trends without assuming linearity or a specific functional form for the
degradation trajectory.

A statistically significant positive $\rho$ indicates ordered progression of
elastic transient morphology over the component lifetime. Across all IMS runs,
we observe strong positive correlations with extremely small $p$-values,
confirming that the latent operator captures time-ordered structural evolution
rather than isolated outliers or random excursions.

\paragraph{Interpretation.}

Both the Health Index and the Time-to-Detection are derived entirely from a
frozen operator trained on interferometric strain transients, without
retraining, calibration, or adaptation to the IMS dataset. Consequently, any
temporal structure observed in $HI(t)$ reflects alignment between the evolving
elastic transient morphology of the mechanical system and the structural
regularities encoded by $\mathcal{F}$, rather than supervised learning,
explicit modeling of bearing dynamics, or retrospective optimization.

This protocol enables a direct and falsifiable evaluation of zero-shot
degradation tracking and early deviation detection in a physically aging
system, forming the central experimental test of cross-domain morphological
sensitivity in this work.
\subsection{IMS Controlled-Split Protocol for Encoder Comparisons}
\label{sec:ims_split_protocol}

The IMS run-to-failure experiments in Sec.~\ref{sec:ims_protocol} use the full trajectories
to quantify drift, early deviation, and health-index evolution. In addition, we introduce
a controlled-split evaluation protocol designed specifically to (i) enable matched
comparisons between different frozen encoders and (ii) isolate regimes where degradation
manifests as intermittent elastic impacts rather than as global spectral drift.

For each selected IMS trajectory, we construct three disjoint record sets:
(i) a nominal reference set $\mathcal{P}_0$ drawn from the early-life segment,
(ii) a held-out nominal test segment, and (iii) a late-life abnormal segment.
In all experiments reported in Sec.~\ref{sec:results_imagenet}, we use
$N_0=120$ records for $\mathcal{P}_0$, and evaluate on $N_{\mathrm{nom}}=250$ held-out nominal
records and $N_{\mathrm{abn}}=250$ late-life records, yielding 620 records per trajectory.
No records from the test segments are used to estimate $\mu_0$, $\Sigma_{\mathrm{LW}}$,
or the threshold $\tau$.

All records are processed using the identical STFT and per-window standardization pipeline
described in Sec.~\ref{sec:tf_preprocessing}. Window-level anomaly scores are computed using
the Mahalanobis distance in latent space (Sec.~\ref{sec:operator_scoring}) and aggregated to
record-level statistics using the $q=0.95$ quantile, denoted $s_{0.95}(t)$.
Performance is summarized using (i) window-level discrimination $\mathrm{AUC}_{\mathrm{win}}$,
(ii) record-level discrimination $\mathrm{AUC}_{\mathrm{rec}}$ computed from $s_{0.95}(t)$,
and (iii) tail enrichment $\Lambda_{\mathrm{tail}}$ under the fixed nominal threshold
$\tau=Q_{0.999}(\mathcal{P}_0)$.

\paragraph{Matched generic baseline (ImageNet).}
To test whether the observed cross-domain sensitivity is specific to the interferometric
morphological prior, we repeat the entire controlled-split protocol using a frozen
EfficientNet-B0 encoder pre-trained on ImageNet. This baseline shares the same architecture,
embedding dimensionality ($d=1280$), preprocessing, anomaly scoring, thresholding, and
aggregation. The only difference is the source of the frozen representation:
(i) interferometric strain transients (morphological operator $\mathcal{F}$) versus
(ii) natural-image supervision (generic baseline).

\paragraph{Physical descriptors (RMS and kurtosis).}
To probe the physical observable driving separation in the intermittent regime, we compute
classical record-level descriptors on the raw time series (channel 0): Root Mean Square (RMS)
as a proxy for signal energy and kurtosis as a proxy for impulsivity. We report correlations
between these descriptors and the record-level anomaly statistic $s_{0.95}$ for both encoders.
\subsection{Statistical Thresholding and Tail Enrichment}
\label{sec:tail_enrichment}

Anomalous behavior is characterized relative to a fixed nominal reference
distribution $\mathcal{P}_0$, constructed exclusively from early-life or
known-normal data and held fixed throughout all experiments. No information
from abnormal conditions enters the estimation of thresholds or scores at any
stage, ensuring a strictly one-sided and physically interpretable protocol.

Rather than posing anomaly detection as a binary classification problem, we
adopt a tail-based characterization aligned with rare-event analyses in
experimental physics, where meaningful deviations appear as excess occupancy in
the extreme tails of a background distribution rather than as shifts in central
tendency. No sequential change-detection schemes (e.g., CUSUM, SPRT, or EWMA) are
required: all results are derived solely from fixed-window anomaly scores,
record-level aggregation, and quantile-based thresholds defined \emph{a priori}
from $\mathcal{P}_0$. Accordingly, $\Lambda_{\mathrm{tail}}$ is interpreted as a
descriptive effect-size measure under a fixed nominal false-alarm convention,
not as a discovery statistic or a claim of universality.

\paragraph{Quantile-based thresholding.}
A detection threshold is defined as a high quantile of the nominal score
distribution,
\begin{equation}
\tau = Q_q(\mathcal{P}_0),
\end{equation}
with $q=0.999$ unless otherwise stated, corresponding to a nominal false-alarm
probability $1-q=10^{-3}$. The threshold is estimated once from $\mathcal{P}_0$
and applied uniformly across all datasets, runs, and experimental conditions.
Finite-sample variability of $\tau$ is negligible compared to the dynamic range
observed under abnormal conditions, and the results are stable under moderate
variations of $q$.

\paragraph{Tail enrichment.}
Given the fixed threshold $\tau$, we define the tail enrichment factor
\begin{equation}
\Lambda_{\mathrm{tail}} =
\frac{
\mathbb{P}\!\left(\mathcal{S}(z)>\tau \mid \mathrm{abnormal}\right)
}{
\mathbb{P}\!\left(\mathcal{S}(z)>\tau \mid \mathrm{nominal}\right)
}.
\end{equation}
By construction, the denominator equals $1-q$, providing a normalized and
physically interpretable baseline. Values
$\Lambda_{\mathrm{tail}}\gg 1$ indicate strong excess occupancy of extreme latent
events relative to nominal operation, while
$\Lambda_{\mathrm{tail}}\approx 1$ indicates no measurable enrichment beyond
statistical expectation. This formulation emphasizes changes in extreme-event
occupancy and is insensitive to shifts in mean score or bulk statistics.

\paragraph{Robustness and physical selectivity.}
Tail enrichment remains stable across high-quantile choices
$q\in[0.995,\,0.9995]$, with $\Lambda_{\mathrm{tail}}$ varying by less than a
factor of two. In contrast, negative-control conditions—including electrically
dominated signals and controlled morphology-destruction transformations—yield
$\Lambda_{\mathrm{tail}}\approx 1$ independently of threshold choice.

As shown in Sec.~\ref{sec:results}, rotating mechanical systems under fault or
advanced degradation exhibit pronounced enrichment
($\Lambda_{\mathrm{tail}}\sim10^{1}$--$10^{2}$), whereas systems lacking coherent
elastic wave propagation do not. This contrast demonstrates that extreme-score
enrichment arises only when structured time--frequency organization associated
with elastic transients is preserved. Accordingly, $\Lambda_{\mathrm{tail}}$
quantifies the response of the fixed operator $\mathcal{F}$ to preserved elastic
transient morphology, rather than serving as a hypothesis-test statistic.

\subsection{Controlled Morphological Degradation}
\label{sec:degradation_protocol}

To determine whether the observed cross-domain anomaly sensitivity reflects a
physically meaningful structural effect rather than a statistical artefact, we
introduce a controlled degradation protocol that selectively disrupts the
\emph{morphological organization} of elastic transients. The central objective is to progressively destroy coherent time--frequency structure while preserving,
up to per-window normalization, low-order amplitude statistics.

If anomaly sensitivity were driven by generic distributional mismatch, signal
energy, or variance-based cues, detection performance would remain largely
invariant under such transformations. Conversely, if sensitivity is governed by
structured elastic transient morphology, performance should degrade in a
selective and transformation-dependent manner. The protocol therefore
constitutes a direct falsification test of the physical hypothesis underlying
the reported zero-shot transfer.

\paragraph{Design principle.}

All transformations are constructed such that, after per-window normalization,
first- and second-order amplitude moments are conserved. As a result, trivial
cues related to signal power, variance, or stationary spectral content are
suppressed. Any systematic loss of anomaly sensitivity must therefore arise
from disruption of coherent time--frequency organization rather than from
changes in marginal statistics.

\paragraph{Transformation families.}

We consider three classes of morphology-altering transformations, each applied
in the time domain prior to time--frequency projection:

\begin{enumerate}
  \item \emph{Spectral low-pass filtering}, which progressively removes
  high-frequency components while preserving integrated signal energy.
  \item \emph{Temporal smearing}, implemented via convolution with a finite-width
  kernel, degrading time localization without altering total spectral power.
  \item \emph{Reverberation-induced self-interference}, implemented through
  multi-tap delay lines that introduce phase dispersion and temporal
  incoherence while preserving spectral support.
\end{enumerate}

All transformed signals are subsequently standardized on a per-window basis,
ensuring matched mean and variance relative to the corresponding untransformed
windows.

\paragraph{Degradation levels.}

For each transformation family, three degradation levels are defined: a control
condition ($L_0$) with no additional morphological perturbation beyond the
intrinsic processing pipeline; a moderate degradation level ($L_1$),
corresponding to partial disruption of transient coherence; and a strong
degradation level ($L_2$), designed to effectively destroy coherent
time--frequency organization through aggressive filtering, smearing, or
phase-dispersive interference.

\paragraph{Falsifiable prediction.}

If anomaly sensitivity is governed by elastic transient morphology, detection
performance should degrade selectively under transformations that disrupt
coherent time--frequency organization, with distinct responses across
transformation families and degradation levels. In contrast, invariance of
performance across all transformations would falsify the physical
interpretation and indicate sensitivity to generic statistical cues. The
results reported in Sec.~\ref{sec:results} directly test this prediction.

\section{Results}
\label{sec:results}
\subsection{Run-to-Failure Degradation and Early Warning (IMS--NASA)}
\label{sec:results_ims}

We first evaluate the frozen latent operator $\mathcal{F}$ on the IMS--NASA
run-to-failure bearing dataset, which captures the full temporal evolution of
mechanical degradation from nominal operation to catastrophic failure. Unlike
static fault-classification benchmarks, this regime probes whether the operator
can track \emph{progressive physical degradation}, identify early deviation
from nominal behavior, and yield an interpretable health indicator under a
strict zero-shot protocol.

Across all runs, anomaly scores are computed relative to a fixed nominal
baseline $\mathcal{P}_0$ derived from the early-life segment of each trajectory.
A detection threshold $\tau$ is defined as the $q=0.999$ quantile of the nominal
distribution, fixing the false-alarm probability \emph{a priori}. Temporal
evolution is characterized using the normalized Health Index
$HI(t)=s_{q}(t)/\tau$, hereafter we use $HI(t)$ with $q=0.99$ unless otherwise stated.

\paragraph{Run 1: Noisy progressive degradation.}

In the first trajectory (IMS 1st\_test), classical separability metrics are weak
(window-level $\mathrm{AUC} \approx 0.55$, record-level
$\mathrm{AUC} \approx 0.62$), indicating that nominal and fault states are not
cleanly separable in a static classification sense. Despite this, the operator
exhibits a clear and statistically significant monotonic drift in extreme-score
statistics. The Health Index increases systematically over time, with a strong
positive Spearman correlation ($\rho \approx 0.56$, $p \ll 10^{-40}$).

Early deviation is detected well before final failure: the exceedance rate
first surpasses $1\%$ at record index 51, and the normalized Health Index crosses
unity at record index 99. The maximum observed severity reaches
$HI_{q99,\max} \approx 14.7$. This behavior indicates gradual structural
degradation rather than a sharp, class-separable transition.

\paragraph{Run 2: Clean degradation regime.}

The second trajectory (IMS 2nd\_test) exhibits a markedly cleaner degradation
pattern. Nominal and abnormal states occupy well-separated regions of latent
space ($\mathrm{AUC}_{\mathrm{win}} \approx 0.92$,
$\mathrm{AUC}_{\mathrm{rec}} \approx 0.99$), but the most salient feature is the
exceptionally early emergence of extreme latent events.

The Health Index shows a strong monotonic increase with time
($\rho \approx 0.68$, $p \ll 10^{-70}$) and crosses the nominal threshold at
record index 11, long before final failure. Tail enrichment is pronounced
($\Lambda_{\mathrm{tail}} \approx 2.5\times10^{2}$), and the maximum severity
reaches $HI_{q99,\max} \approx 20.5$, indicating sustained accumulation of
extreme morphological deviations without saturation.

\paragraph{Run 3: Rapid-onset failure.}

The third trajectory (IMS 3rd\_test) displays a rapid-onset degradation regime.
The Health Index remains stable throughout nominal operation before undergoing
a sharp transition during the late-life phase. Separability is high
($\mathrm{AUC}_{\mathrm{win}} \approx 0.90$,
$\mathrm{AUC}_{\mathrm{rec}} \approx 0.98$), and extreme-event occupancy
increases abruptly, yielding a large tail enrichment
($\Lambda_{\mathrm{tail}} \approx 8.6\times10^{2}$).

The Health Index rapidly exceeds unity and reaches values $HI_{q99} \gg 10$,
signaling imminent failure with high confidence. This trajectory is consistent
with a regime characterized by prolonged stable operation followed by a rapid
structural collapse.

\paragraph{Interpretation across runs.}

Across all three trajectories, the latent operator consistently captures
monotonic drift in extreme latent events, despite substantial variation in
static separability metrics. This demonstrates that window-level or record-level
AUC is not the primary indicator of physical sensitivity in run-to-failure
settings. Instead, degradation manifests as a progressive enrichment of rare,
structurally anomalous transients.

These results establish the IMS--NASA experiments as the central physical
evidence of cross-domain transfer: a representation trained exclusively on
interferometric strain transients functions as a zero-shot morphological
operator capable of tracking mechanical degradation and providing early warning
in a real run-to-failure system.
\subsection{Zero-Shot Transfer in Static Fault Regimes (CWRU and VSB)}
\label{sec:results_zeroshot}

Having established that the latent operator $\mathcal{F}$ captures progressive
degradation and early deviation in run-to-failure systems (IMS--NASA), we next
evaluate its behavior in \emph{static fault regimes}, where mechanical states
are discretely labeled but do not evolve continuously in time. This setting
serves as a complementary validation of cross-domain transfer, testing whether
preserved elastic transient morphology also yields separability under
controlled, steady-state fault conditions.

The operator is trained exclusively on interferometric strain transients and is
applied without modification in a strict zero-shot setting.
\begin{figure}[t]
  \centering
  \includegraphics[width=0.90\linewidth]{figures/Fig3.pdf}
  \caption{
  \textbf{Zero-shot performance and physical specificity.}
  Receiver operating characteristic (ROC) curves for (a) the CWRU rotating
  machinery dataset (mechanical domain) and (b) the VSB electrical control
  dataset (electromagnetic domain).
  Strong discrimination is observed only in mechanically governed systems where
  elastic transient morphology is preserved, while performance collapses toward
  chance in electrically dominated signals, establishing a clear physical
  boundary for zero-shot transfer.
  }
  \label{fig:roc}
\end{figure}

Figure~\ref{fig:roc}(a) reports receiver operating characteristic (ROC) curves
for mechanically induced fault conditions in the CWRU bearing dataset. Strong
discrimination is observed, with window-level predictions yielding
$\mathrm{AUC}_{\mathrm{win}} \approx 0.90$ and file-level aggregation producing
near-perfect separability,
$\mathrm{AUC}_{\mathrm{file}} \approx 0.99$. 

The improvement from window-level to file-level discrimination indicates that
anomalous morphology is not confined to isolated transient windows, but recurs
coherently across multiple realizations within a recording. This persistence is
consistent with the behavior observed in the IMS run-to-failure experiments,
where degradation manifests as repeated excursions into the extreme tail of the
latent distribution rather than as isolated outliers.

As a negative control, the same frozen operator is applied to electrically
dominated vibration signals from the VSB dataset, which are characterized by
narrowband electromagnetic interference and incoherent sensor noise rather than
mechanically mediated elastic transients. As shown in
Fig.~\ref{fig:roc}(b), discrimination collapses toward chance in this regime,
with no measurable tail enrichment
($\Lambda_{\mathrm{tail}} \approx 1$ at the nominal false-alarm rate).

The sharp contrast between mechanically governed and electrically dominated
systems establishes a clear physical boundary for zero-shot transfer. High
anomaly sensitivity emerges only in systems governed by elastic wave
propagation in constrained media, while systems lacking such propagation
exhibit weak and non-selective responses. This behavior confirms that the latent
operator does not respond generically to non-Gaussian statistics, signal
energy, or distributional shifts, but instead requires coherent elastic
organization to produce separability.

\paragraph{Separation across physical scales.}

For context, source-domain interferometric transients correspond to
dimensionless strain amplitudes of order $h \sim 10^{-21}$ at characteristic
frequencies of $\mathcal{O}(10^{2})\,\mathrm{Hz}$. In contrast, bearing faults in
the CWRU dataset involve localized mechanical impacts producing accelerations
of $\mathcal{O}(10^{1})\,g$ at characteristic frequencies of
$\mathcal{O}(10^{3})\,\mathrm{Hz}$. The observed zero-shot separability therefore spans radically different sensing modalities,
physical units, and characteristic frequency ranges, reinforcing that anomaly sensitivity is
not tied to absolute scale or sensor modality, but to preserved organization in elastic
transient morphology.
\subsection{Falsification via a Generic Feature Baseline (ImageNet) on IMS}
\label{sec:results_imagenet}

A central question is whether the observed anomaly sensitivity reflects a specific prior
for elastic transient morphology learned from interferometric instrumentation, or whether
any high-capacity convolutional encoder would produce similar behavior under the same
pipeline. To falsify the ``generic CNN features'' hypothesis, we repeat the IMS analysis
under the controlled-split protocol defined in Sec.~\ref{sec:ims_split_protocol}, comparing
the interferometric operator with a matched EfficientNet-B0 encoder pre-trained on ImageNet.

Table~\ref{tab:ims_imagenet} summarizes discrimination and extreme-event statistics for the
selected IMS trajectories. In the clean degradation regime (2nd\_test), both encoders reach
high record-level separability, consistent with faults that induce strong global structure.
However, in the intermittent regime (IMS 4th\_test), where degradation manifests as sparse impacts
embedded in a noisy background, the generic baseline collapses: tail enrichment remains near
the nominal expectation ($\Lambda_{\mathrm{tail}}\approx 2$), while the interferometric operator
exhibits massive extreme-event amplification ($\Lambda_{\mathrm{tail}}\approx 860$) under the
same fixed false-alarm rate.

\begin{table}[t]
\caption{\label{tab:ims_imagenet}
IMS controlled-split evaluation (Sec.~\ref{sec:ims_split_protocol}) comparing the frozen
interferometric morphological operator and a matched ImageNet-pretrained baseline.
Record-level metrics use the $q=0.95$ score quantile $s_{0.95}$; the tail threshold is fixed
as $\tau = Q_{0.999}(\mathcal{P}_0)$.
}
\begin{ruledtabular}
\begin{tabular}{lcccccc}
& \multicolumn{3}{c}{Morphological (LIGO)} & \multicolumn{3}{c}{Generic (ImageNet)} \\
\cline{2-4}\cline{5-7}
IMS run
& $\mathrm{AUC}_{\mathrm{win}}$
& $\mathrm{AUC}_{\mathrm{rec}}$
& $\Lambda_{\mathrm{tail}}$
& $\mathrm{AUC}_{\mathrm{win}}$
& $\mathrm{AUC}_{\mathrm{rec}}$
& $\Lambda_{\mathrm{tail}}$ \\
\hline
1st\_test
& 0.616 & 0.702 & 1.27
& 0.543 & 0.538 & 1.60 \\
2nd\_test
& 0.999 & 1.000 & 191.6
& 0.921 & 0.997 & 19.1 \\
4th\_test
& 0.901 & 0.982 & \textbf{859.5}
& 0.653 & 0.652 & \textbf{1.97} \\
\end{tabular}
\end{ruledtabular}
\end{table}

This dissociation indicates that sensitivity to weak or intermittent elastic anomalies is
not a generic property of CNN feature extraction, but depends on the latent geometry induced
by training on interferometric non-Gaussian transients.

\subsection{Mechanism: Enhanced Coupling to Impulsivity (Kurtosis) in the Intermittent Regime}
\label{sec:results_kurtosis}

To clarify why the interferometric operator outperforms the generic baseline in the
intermittent regime, we analyze the coupling between the record-level anomaly statistic
$s_{0.95}$ and classical physical descriptors computed on the raw time series: RMS amplitude
(proxy for signal energy) and kurtosis (proxy for impulsivity).

Table~\ref{tab:kurtosis_rms} reports record-level Pearson correlations for  IMS 4th\_test under the
controlled-split protocol. While both representations correlate with RMS, the interferometric
operator exhibits stronger coupling to kurtosis, consistent with higher sensitivity to sparse,
impulsive elastic impacts embedded in a noisy background. This enhanced impulsivity coupling
explains the divergence in extreme-event statistics: the generic baseline fails to enrich the
tail until anomalies alter global texture, whereas the interferometric operator selectively
amplifies rare impulsive transients.

\begin{table}[t]
\caption{\label{tab:kurtosis_rms}
Record-level Pearson correlations between the aggregated anomaly statistic $s_{0.95}$ and
physical descriptors in the intermittent regime (IMS 4th\_test).
}
\begin{ruledtabular}
\begin{tabular}{lcc}
Representation & Corr.\ w/ Kurtosis & Corr.\ w/ RMS \\
\hline
Morphological (LIGO) & 0.7789 & 0.8998 \\
Generic (ImageNet) & 0.6829 & 0.7520 \\
\end{tabular}
\end{ruledtabular}
\end{table}
\subsection{Controlled Morphological Degradation}

To determine whether cross-domain anomaly sensitivity arises from preserved
time--frequency organization rather than from marginal statistics or signal
energy, we evaluate the frozen operator $\mathcal{F}$ under a set of controlled
morphological perturbations. Detection performance is assessed using the
protocol defined in Sec.~\ref{sec:degradation_protocol}, which selectively
disrupts transient structure via (i) spectral low-pass filtering,
(ii) temporal smearing, and (iii) reverberation-induced self-interference.

All transformed signals are renormalized on a per-window basis, ensuring that
any change in performance cannot be attributed to shifts in first- or
second-order amplitude statistics. These transformations are not used for data
augmentation or model regularization, but solely as controlled physical
interventions designed to probe the operator’s sensitivity to coherent
time--frequency organization.

\begin{figure}[b]
  \centering
  \includegraphics[width=\columnwidth]{figures/Fig4.pdf}
  \caption{
  \textbf{Falsification via controlled morphological destruction.}
  Zero-shot detection performance, quantified by window-level $\mathrm{AUC}$,
  under transformations that selectively degrade coherent time--frequency
  structure while preserving marginal amplitude statistics.
  Selective and transformation-dependent performance degradation indicates
  sensitivity to elastic transient morphology rather than to signal energy or
  stationary statistics.
  }
  \label{fig:morph}
\end{figure}

Figure~\ref{fig:morph} reports the window-level area under the ROC curve
($\mathrm{AUC}_{\mathrm{win}}$) as a function of degradation level for each
transformation family. While $\mathrm{AUC}$ is used here as a convenient
summary statistic, the emphasis is on relative trends across transformations
rather than on absolute classification performance.

The response of $\mathcal{F}$ depends strongly on the physical nature of the
morphological disruption. Spectral low-pass filtering produces a monotonic
collapse in anomaly sensitivity as the cutoff frequency is reduced. This trend
reflects the progressive removal of high-frequency components that support
sharp transient localization, broadband interference, and elastic mode
coupling—features central to the morphology encoded in the latent space.

Temporal smearing induces a more moderate degradation. Although transient
localization in time is weakened, the global organization of energy across
frequency bands remains partially intact, allowing $\mathcal{F}$ to retain a
substantial fraction of its discriminative power. This behavior indicates that
the operator is not narrowly tuned to impulsive events, but responds to
extended, yet coherent, elastic structure.

In contrast, reverberation does not produce a monotonic loss of sensitivity.
Under reverberation, the operator maintains
$\mathrm{AUC}_{\mathrm{win}} \approx 0.90$ even at strong degradation levels,
whereas low-pass filtering reduces
$\mathrm{AUC}_{\mathrm{win}}$ to approximately $0.62$.
Reverberation redistributes coherent energy across time through delayed
self-interference while preserving spectral support, indicating that the
operator responds to the presence and organization of structured elastic
transients rather than to precise temporal alignment or isolated high-amplitude
peaks.

Taken together, these perturbation experiments constitute a direct and
falsifiable test of the underlying physical hypothesis. If anomaly sensitivity
were driven by generic statistical cues, signal energy, or
architecture-dependent biases, performance would remain approximately invariant
under amplitude-preserving transformations. Instead, separability degrades
selectively in a transformation-dependent manner that tracks the preservation
or destruction of coherent elastic transient morphology. This behavior supports
the interpretation that cross-domain anomaly sensitivity is governed by
structured elastic wave phenomena rather than by distributional or
energy-based effects.
\subsection{Physical Interpretability and Comparison with Reconstruction-Based Baselines}
\label{sec:results_physical}

To assess whether the observed cross-domain anomaly sensitivity reflects
physically meaningful structure rather than generic statistical effects, we
analyze the anomaly scores produced by the latent operator $\mathcal{F}$ and
compare them with those obtained from a reconstruction-based convolutional
autoencoder (ConvAE) baseline. The purpose of this comparison is not to rank
models by detection performance, but to use each approach as a diagnostic
instrument probing which physical observables dominate their respective anomaly
responses.

\begin{figure}[t]
  \centering
  \includegraphics[width=\columnwidth]{figures/Fig5.pdf}
  \caption{
  \textbf{Physical interpretability of anomaly scores.}
  Pearson correlation between anomaly scores and classical signal descriptors.
  Reconstruction-based scores (ConvAE) exhibit stronger coupling to amplitude-
  related statistics, whereas the morphological operator $\mathcal{F}$ correlates
  primarily with Effective Bandwidth and Spectral Entropy and remains largely
  orthogonal to signal energy.
  }
  \label{fig:physcorr}
\end{figure}

Both operators are evaluated on identical target-domain data under the same
preprocessing and scoring pipeline. Their window-level anomaly scores are
correlated with a set of classical signal descriptors commonly used in vibration
analysis, including amplitude-based measures (e.g., RMS and peak value) and
structure-related observables (e.g., effective bandwidth and spectral entropy).
Figure~\ref{fig:physcorr} reports the resulting Pearson correlation
coefficients.

Under the present protocol, ConvAE anomaly scores exhibit non-negligible
correlations with amplitude-related descriptors and broader, less selective
correlations with higher-order spectral measures. This behavior is consistent
with reconstruction error acting as a generic measure of deviation from the
training manifold, implicitly coupling sensitivity to a mixture of signal
energy, variance, and spectral content.

In contrast, anomaly scores produced by the latent operator $\mathcal{F}$ are
nearly orthogonal to amplitude-based measures, including RMS and peak value,
while exhibiting their strongest correlations with effective bandwidth
($r \approx 0.63$) and spectral entropy ($r \approx 0.49$). These observables
quantify the redistribution and dispersion of energy across the
time--frequency plane and serve as proxies for the coherence and complexity of
elastic wave packets, independent of absolute signal scale or sensor modality.

This correlation structure indicates that $\mathcal{F}$ responds preferentially
to morphological decoherence—i.e., disruption of structured time--frequency
organization—rather than to increases in signal power or variance. In physical
terms, the operator is sensitive to \emph{how} energy is organized across time
and frequency, not to \emph{how much} energy is present.

The contrasting correlation patterns of the ConvAE and $\mathcal{F}$ therefore
highlight a fundamental difference in the information encoded by the two
approaches. Reconstruction-based methods remain implicitly tied to the energy
and variance structure of the training data, while the latent operator
$\mathcal{F}$ isolates structural regularities associated with elastic wave
propagation in constrained media. This decoupling from amplitude-driven
detection provides a physically grounded interpretation of the selective
zero-shot transfer observed in the preceding sections.
\FloatBarrier

\subsection{Tail Enrichment of Extreme Morphological Events}

Beyond aggregate discrimination metrics, we characterize anomaly sensitivity
through the statistics of extreme events in the latent space. This analysis is
deliberately aligned with rare-event detection paradigms in experimental
physics, including gravitational-wave instrumentation, where physically
meaningful signals are identified not by shifts in average behavior but by
excess occupancy in the extreme tails of background distributions.

Rather than interpreting anomaly scores through their mean or variance, we
analyze exceedance statistics relative to a fixed reference distribution
constructed from nominal operation. Figure~\ref{fig:tail} shows the probability
density of window-level anomaly scores $\mathcal{S}$ under nominal and
fault-induced mechanical conditions for the reverberation condition $L_1$.

\begin{figure}[t]
  \centering
  \includegraphics[width=0.98\linewidth]{figures/Fig6.pdf}
  \caption{
  \textbf{Tail enrichment of morphological anomaly scores in the latent space.}
  Log-scale probability density of \emph{window-level} anomaly scores $\mathcal{S}$
  for nominal (gray) and fault-induced (black) mechanical states under the
  reverberation condition $L_1$.
  The detection threshold $\tau$ is defined as the $q=0.999$ quantile of the
  nominal distribution, corresponding to a nominal false-alarm rate of
  $1-q=10^{-3}$.
  Densities are estimated using logarithmically spaced histogram bins.
  Beyond this threshold, abnormal windows exhibit a pronounced excess of extreme
  events, with exceedance probability
  $\mathbb{P}(\mathcal{S}>\tau \mid \mathrm{abnormal})=6.49\times10^{-2}$,
  yielding a tail enrichment factor
  $\Lambda_{\mathrm{tail}}=64.9$ (95\% CI: [55.7, 74.5]).
  The separation is dominated by rare, high-score excursions rather than by
  shifts in central tendency.
  }
  \label{fig:tail}
\end{figure}

While the nominal anomaly-score distribution exhibits a rapidly decaying tail,
fault-induced mechanical states display a pronounced excess of extreme latent
events. At a fixed nominal false-alarm rate of $10^{-3}$, defined by the
$q=0.999$ quantile of the nominal distribution, we observe a tail enrichment
factor $\Lambda_{\mathrm{tail}}=64.9$, with a nonparametric bootstrap 95\%
confidence interval of [55.7, 74.5]. The observed nominal exceedance rate
($1.01\times10^{-3}$) is consistent with the target false-alarm rate within
finite-sample uncertainty.

Crucially, this separation is driven by rare, high-score excursions rather than
by a global shift of the anomaly-score distribution. In physical terms, the
latent operator $\mathcal{F}$ responds to the emergence of transient windows
exhibiting extreme morphological decoherence—i.e., strong disruption of
structured time--frequency organization—analogous to excess-power or burst-like
detection strategies in interferometric gravitational-wave searches.

Negative-control conditions, including electrically dominated signals and
selected morphology-destruction transformations, do not exhibit tail occupancy
above the nominal false-alarm rate. The absence of enrichment in these regimes
confirms that the observed heavy-tail behavior is not a generic consequence of
distributional mismatch, score rescaling, or non-Gaussianity per se, but arises
specifically from preserved elastic transient morphology in mechanically
governed systems.

We emphasize that $\Lambda_{\mathrm{tail}}$ is not interpreted as a discovery
statistic and does not constitute a hypothesis test. Rather, it quantifies the
degree to which fault-induced mechanical states populate the extreme-event
regime of the latent space relative to nominal operation. As such,
$\Lambda_{\mathrm{tail}}$ provides a physically grounded characterization of
anomaly sensitivity consistent with the interpretation of $\mathcal{F}$ as a
morphological detector of rare, structured elastic transients.

\section{Discussion}
\label{sec:discussion}

\subsection{Resolution-Dependent Morphological Sensitivity and Conditional Invariance}

The term ``morphological invariance'' is used here only as shorthand for a
resolution-dependent stability of the operator response, not as a fundamental
invariance of the underlying dynamics. It refers to the behavior of a fixed
latent operator under changes of physical system, rather than to an invariance
of the equations of motion. The observed behavior should therefore be
interpreted as a representation-induced property of $\mathcal{F}$ under a
specific experimental protocol, not as evidence for a universal symmetry,
conserved quantity, or scale-free law of elastic dynamics.

The scope of this work is deliberately restricted to systems governed by
mechanically mediated elastic wave propagation in constrained solids. We do not
attempt to optimize or validate the approach in regimes dominated by stochastic
acoustic or electromagnetic processes.

The time--frequency resolution employed is chosen to resolve short-duration,
broadband elastic transients while suppressing explicit dependence on absolute
scale, sensor modality, and physical units through per-window normalization. At
this resolution, elastic wave packets remain identifiable as structured objects
in the time--frequency plane. Within this regime, anomaly sensitivity transfers
across physically unrelated systems only when coherent elastic transient
morphology is preserved.

Crucially, this transfer occurs without retraining, fine-tuning, domain
adaptation, or access to target-domain semantics. In the IMS and CWRU bearing
experiments, the operator $\mathcal{F}$—trained exclusively on interferometric
strain transients—exhibits monotonic drift detection, early time-to-detection,
and pronounced tail enrichment under mechanical fault conditions, even when
classical classification metrics such as window-level AUC vary substantially.
This indicates that $\mathcal{F}$ is not encoding system-specific operating
conditions, but is responding to structural properties intrinsic to elastic wave
propagation in constrained media.

\subsection{Domain of Validity and Physical Selectivity}

A central implication of these results is that the anomaly sensitivity exhibited
by $\mathcal{F}$ is not universal, but sharply conditional on the physical
mechanisms governing signal generation and propagation. The systematic contrast
observed between mechanically mediated systems and electrically dominated
signals provides an explicit, experimentally grounded delineation of the domain
in which the proposed approach is valid.

When applied to electrically dominated vibration signals, anomaly sensitivity
collapses toward weak and non-selective behavior. Although window-level AUC
values may remain marginally above chance, they lack the defining signatures
observed in mechanically governed systems: stability under aggregation,
monotonic temporal drift, and enrichment of extreme latent events. This collapse
occurs despite the presence of non-Gaussian statistics, demonstrating that
non-Gaussianity alone is insufficient to induce cross-domain anomaly sensitivity.

These observations indicate that $\mathcal{F}$ does not respond to generic
distributional shifts, broadband noise, or narrowband electromagnetic
interference. Instead, measurable sensitivity emerges only in systems that
support mechanically mediated elastic wave propagation in constrained media.
The absence of strong anomaly sensitivity in electrically dominated systems
therefore functions as a negative control, reinforcing the physical selectivity
of the operator rather than indicating a methodological limitation.

We do not claim that $\mathcal{F}$ provides a first-principles representation of
elastic dynamics. Rather, it functions as an empirical measurement instrument
whose response reveals structural regularities when elastic wave propagation is
coherent and constrained.
\subsection{Physical Origin of Morphological Regularities}

The conditional nature of the observed transfer admits a direct physical
interpretation. Both interferometric instrumentation and rotating mechanical
assemblies support elastic wave propagation in bounded, structured media. In
interferometers, transient disturbances excite elastic modes of suspended
optical components and support structures; in rolling-element bearings, impacts
and frictional contacts excite guided elastic waves in raceways, rolling
elements, shafts, and housings.

The physical interpretation advanced here is operational: $\mathcal{F}$ acts as
a morphology-sensitive instrument under a specified preprocessing and scoring
protocol, and its selectivity is established empirically through negative
controls and controlled morphology-destruction interventions.

Recent work has shown that deep representations induce highly structured latent
geometries reflecting both architecture and training data
\cite{balestriero2021geometry}, while geometric deep learning highlights how
inductive biases tied to data structure can dominate downstream behavior even
under frozen, zero-shot deployment \cite{bronstein2021geometric}.

In both interferometric and mechanical systems, measured signals reflect the
superposition, dispersion, and reflection of elastic wave packets constrained
by geometry, material properties, and boundary conditions. These processes
impose characteristic time--frequency organizations that are largely independent
of absolute physical scale, yet highly sensitive to coherence and confinement.
The empirical results indicate that $\mathcal{F}$ is selectively sensitive to
these regularities: when elastic propagation remains coherent, anomaly
sensitivity, drift, and tail enrichment emerge; when this organization is
disrupted, sensitivity degrades in a predictable, transformation-dependent
manner.
\subsection{Structural Sensitivity Beyond Energy-Based Detection}

The comparative behavior between $\mathcal{F}$ and reconstruction-based
baselines admits a clear physical interpretation. Reconstruction-based anomaly
detectors implicitly learn scale-dependent statistics of the training manifold,
flagging deviations when test signals differ in amplitude, variance, or spectral
power distribution. As a result, reconstruction error remains intrinsically
coupled to energy-based cues.

In contrast, the latent operator $\mathcal{F}$ encodes structural properties
associated with elastic wave propagation in constrained media. Its anomaly
sensitivity is governed not by absolute signal power, but by the organization of
energy across time and frequency, reflecting the coherence and dispersion of
propagating elastic wave packets. This decoupling from amplitude-driven cues
explains why anomaly sensitivity can be preserved across domains with radically
different sensing modalities, physical units, and energy scales.

The response of $\mathcal{F}$ to controlled morphological degradation is
therefore naturally interpreted as a loss of coherent time--frequency
organization. Dispersion, temporal smearing, and reverberation progressively
disrupt elastic transient structure, leading to a breakdown of the patterns to
which the latent representation is tuned.

\subsection{Broader Physical Scope}

Within its physically delimited domain of validity, the results suggest a
broader implication: representations developed in precision physics contexts
may encode structural priors that reflect general properties of wave-mediated
dynamics in bounded media, rather than features tied to a specific instrument or
dataset.

Under appropriate physical conditions, such representations can function as
morphology-sensitive measurement devices rather than task-specific classifiers.
The present results indicate that cross-domain transfer of anomaly sensitivity
is physically mediated rather than algorithmic, and should therefore be expected
only when underlying propagation mechanisms and morphological constraints are
shared.
\section{Conclusion}
\label{sec:conclusion}

We have shown that anomaly sensitivity can transfer across physically unrelated
systems when their measured signals preserve coherent elastic transient
morphology. A latent operator trained exclusively on interferometric strain
transients from gravitational-wave detector instrumentation exhibits robust
zero-shot sensitivity when applied to rotating mechanical systems governed by
elastic wave propagation, without retraining, fine-tuning, or domain adaptation.

The observed transfer is selective rather than generic. Pronounced anomaly
sensitivity, monotonic drift, early time-to-detection, and enrichment of extreme
latent events arise only in systems that support mechanically mediated elastic
wave propagation in constrained media. In contrast, electrically dominated
signals and controlled morphology-destruction transformations do not exhibit
these signatures, thereby defining a clear and physically motivated domain of
validity. These negative results rule out explanations based on signal energy,
marginal statistics, or generic distributional mismatch.

In run-to-failure regimes, the fixed operator yields a normalized and
interpretable Health Index that tracks progressive degradation and identifies
deviation from nominal operation well before catastrophic failure. This behavior
persists even when classical classification metrics vary across runs,
demonstrating sensitivity to structural evolution rather than to discrete fault
labels.

These findings support an empirically observed, resolution-dependent morphological sensitivity, which can be described as a conditional invariance of the operator response under changes of physical system. Within the stated experimental protocol, the operator $\mathcal{F}$ functions as a morphology-sensitive
measurement device rather than as a task-specific classifier. More broadly, this
work illustrates how representations developed for precision measurement in
fundamental physics can encode transferable structural priors, enabling
physically selective zero-shot generalization across systems that share common
elastic propagation phenomenology.
\section*{Data and Code Availability}

All raw target-domain sensor datasets used in this work are publicly available
from their original providers (IMS--NASA, CWRU, and VSB) and are not redistributed
here. Source-domain interferometric training data and trained model weights are
not publicly released.

Figure-level computational reproducibility is provided through derived,
non-invertible artifacts and deterministic scripts that regenerate all figures
and numerical values reported in the manuscript. Full implementation, training,
and preprocessing details are provided in the Supplemental Material.

Repository: \url{https://github.com/joseandreuu/aether-morphology-paper}

\bibliographystyle{apsrev4-2}
\bibliography{references}
\end{document}